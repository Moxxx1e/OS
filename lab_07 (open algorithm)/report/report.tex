\documentclass[a4paper,12pt]{article} % тип документа
% report, book

%  Русский язык

\usepackage[T2A]{fontenc}			% кодировка
\usepackage[utf8]{inputenc}			% кодировка исходного текста
\usepackage[english,russian]{babel}	% локализация и переносы
\usepackage{listings}                % листинги
\usepackage{color}
\usepackage{amssymb}                 % пустое множество


\definecolor{mygreen}{rgb}{0,0.6,0}
\definecolor{mygray}{rgb}{0.5,0.5,0.5}
\definecolor{mymauve}{rgb}{0.58,0,0.82}

\lstset{ 
  backgroundcolor=\color{white},   % choose the background color; you must add \usepackage{color} or \usepackage{xcolor}; should come as last argument
  basicstyle=\footnotesize,        % the size of the fonts that are used for the code
  breakatwhitespace=false,         % sets if automatic breaks should only happen at whitespace
  breaklines=true,                 % sets automatic line breaking
  captionpos=b,                    % sets the caption-position to bottom
  commentstyle=\color{mygreen},    % comment style
  deletekeywords={...},            % if you want to delete keywords from the given language
  escapeinside={\%*}{*)},          % if you want to add LaTeX within your code
  extendedchars=true,              % lets you use non-ASCII characters; for 8-bits encodings only, does not work with UTF-8
  firstnumber=1,                % start line enumeration with line 1000
  frame=single,	                   % adds a frame around the code
  keepspaces=true,                 % keeps spaces in text, useful for keeping indentation of code (possibly needs columns=flexible)
  keywordstyle=\color{blue},       % keyword style
  language=Python,                 % the language of the code
  morekeywords={*,...},            % if you want to add more keywords to the set
  numbers=left,                    % where to put the line-numbers; possible values are (none, left, right)
  numbersep=5pt,                   % how far the line-numbers are from the code
  numberstyle=\tiny\color{mygray}, % the style that is used for the line-numbers
  rulecolor=\color{black},         % if not set, the frame-color may be changed on line-breaks within not-black text (e.g. comments (green here))
  showspaces=false,                % show spaces everywhere adding particular underscores; it overrides 'showstringspaces'
  showstringspaces=false,          % underline spaces within strings only
  showtabs=false,                  % show tabs within strings adding particular underscores
  stepnumber=1,                    % the step between two line-numbers. If it's 1, each line will be numbered
  stringstyle=\color{mymauve},     % string literal style
  tabsize=2	                   % sets default tabsize to 2 spaces
}


% Картинки
\usepackage{graphicx}
\graphicspath{{img/}}
\DeclareGraphicsExtensions{.pdf,.png,.jpg}

% Математика
\usepackage{amsmath,amsfonts,amssymb,amsthm,mathtools} 
% Знак множества
\usepackage{wasysym}
\usepackage{pdfpages}

\usepackage{geometry}
\geometry{left=2cm}
\geometry{right=1.5cm}
\geometry{top=1cm}
\geometry{bottom=2cm}
\include{references}

\begin{document} % начало документа
\begin{figure}[h!]
	\begin{center}
		{\includegraphics[width = \textwidth]{titul.eps}}
	\end{center}
\end{figure}

%\vspace*{10mm}

\huge
\begin{center}
	Лабораторная работа №7
\end{center}

\vspace*{10mm}

\LARGE
\begin{center}
	Дисциплина: <<Операционные системы>>
\end{center}

\vspace*{20mm}

\large
\begin{flushleft}
	Студент: Колганов О.C. \\
	Группа: ИУ7-62Б \\
	Оценка (баллы): \\
	Преподаватель: Рязанова Н.Ю.
\end{flushleft}

\vspace*{50mm}

\large
\begin{center}
	Москва, 2020 г.
\end{center}

\thispagestyle{empty}

\newpage
\textbf{Задание.}

Построить схему выполнения системного вызова open() в зависимости отзначения основных флагов определяющих открытие файла на чтение, назапись, на выполнение и на создание нового файла. В схеме должны быть названия функций и кратко указаны выполняемые ими действия. По ГОСТ уэто делается с помощью выносных линий в фигурных скобках. 
В схему нужно обязательно включить следующие действия, выполняемыесоответствующими функциями ядра:
\begin{itemize}
\item[1)] копирование названия файла из пространства пользователя впространство ядра;
\item[2)] блокировка/разблокировка (spinlock) структуры $files \textunderscore struct$ и других действий в разных функциях;
\item[3)] алгоритм поиска свободного дескриптора открытого файла;
\item[4)] работу со структурой nameidata – инициализация ее полей;
\item[5)] алгоритм разбора пути (кратко);
\item[6)] инициализацию полей struct file;
\item[7)] «открытие» файла для чтения, записи или выполнения;
\item[8)] создание inode в случае отсутствия открываемого файла.
\end{itemize}


\newpage


\begin{figure}[h!]
\centering
\includegraphics[width=1\textwidth]{lab_07.eps}
\end{figure}


\newpage


\begin{figure}[h!]
\centering
\includegraphics[width=1\textwidth]{lab_07_3.eps}
\end{figure}

\begin{figure}[h!]
\centering
\includegraphics[width=1\textwidth]{lab_07_4.eps}
\end{figure}

\begin{figure}[h!]
\centering
\includegraphics[width=1\textwidth]{lab_07_5.eps}
\end{figure}

\begin{figure}[h!]
\centering
\includegraphics[width=1\textwidth]{lab_07_6.eps}
\end{figure}

\begin{figure}[h!]
\centering
\includegraphics[width=1\textwidth]{lab_07_7.eps}
\end{figure}

\begin{figure}[h!]
\centering
\includegraphics[width=1\textwidth]{lab_07_8.eps}
\end{figure}

\begin{figure}[h!]
\centering
\includegraphics[width=0.5\textwidth]{lab_07_end.eps}
\end{figure}


\end{document}
